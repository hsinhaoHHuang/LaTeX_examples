
\documentclass[12pt, a4paper]{article} % declares the document type, class

\usepackage{CJK}                                           % for Chinese
\usepackage{graphicx}
\graphicspath{ {images/} }                                 %configuring the graphicx package
\usepackage[dvipsnames]{xcolor}
%\usepackage[ total={6.5in,8.75in}, top=1.2in, left=0.9in, includefoot ]{geometry}

\definecolor{mypink1}{rgb}{0.858, 0.188, 0.478}
\definecolor{mypink2}{RGB}{219, 48, 122}
\definecolor{mypink3}{cmyk}{0, 0.7808, 0.4429, 0.1412}
\definecolor{mygray}{gray}{0.6}

\title{My first LaTeX document}                            % the document title
\author{Hsinhao Huang\thanks{hsinaohuang@gmail.com}}       % here you write the name of the author(s)
\date{\today}                                              % or enter the date manually

\begin{document} % the body of the document

    \maketitle

    \tableofcontents

    \begin{abstract}
        This is a simple paragraph at the beginning of the
        document. A brief introduction about the main subject.
    \end{abstract}


    \section{Basic section}
        After our abstract we can begin the first paragraph, then press ``enter'' twice to start the second one.

        This line will start a second paragraph.

        I will start the third paragraph and then add \\ a manual line break which causes
        this text to start on a new line but remains part of the same paragraph.
        Alternatively, I can use the \verb|\newline|\newline command to start a new line,
        which is also part of the same paragraph.

        First document. This is a simple example, with no
        extra parameters or packages included.

    \section{Emphasis}
        Some of the \textbf{greatest}                          % bold text in LaTeX is typeset using the \textbf{...} command.
        discoveries in \underline{science}                     % italicised text is produced using the \textit{...} command.
        were made by \textbf{\textit{accident}}.               %  to underline text use the \underline{...} command.

        Some of the greatest \emph{discoveries} in science
        were made by accident.

        \textit{Some of the greatest \emph{discoveries}        % Inside normal text, the emphasized text is italicized, but this behaviour is reversed if used inside an italicized text
        in science were made by accident.}

        \textbf{Some of the greatest \emph{discoveries}
        in science were made by accident.}

    \section{Figures}
        \begin{figure}[h]
            \centering
            \includegraphics[width=0.75\textwidth]{fig_test}   % set the figure’s width to 75% of the text width
            \caption
            {
                A nice plot.
            }
            \label{fig:test}
        \end{figure}

        As you can see in figure \ref{fig:test},
        the function grows near the origin.
        This example is on page \pageref{fig:test}.

    \section{Lists}
        \begin{itemize}                                        % unordered list
            \item The individual entries are indicated with
                  a black dot, a so-called bullet.
            \item The text in the entries may be of any length.
        \end{itemize}

        \begin{enumerate}                                      % ordered list
            \item This is the first entry in our list.
            \item The list numbers increase with each entry we add.
        \end{enumerate}

        Change the labels using \verb|\item[label text]| in an \texttt{itemize} environment
        \begin{itemize}
            \item This is my first point
            \item Another point I want to make
            \item[!] A point to exclaim something!
            \item[NOTE] This entry has no bullet
            \item[] A blank label?
        \end{itemize}

        \begin{enumerate}
            \item First level item
            \item First level item

            \begin{enumerate}
                \item Second level item
                \item Second level item

                \begin{enumerate}
                    \item Third level item
                    \item Third level item

                    \begin{enumerate}
                        \item Fourth level item
                        \item Fourth level item
                    \end{enumerate}
                \end{enumerate}
            \end{enumerate}
        \end{enumerate}

    \section{Math}
        \begin{math}                                           % inline equation
            E=mc^2
        \end{math} is typeset in a paragraph using inline math mode---as is $E=mc^2$, and so too is \(E=mc^2\).

        The mass-energy equivalence is described by the famous equation
        \[ E=mc^2 \] discovered in 1905 by Albert Einstein \cite{Einstein1905AnP}.

        In natural units ($c = 1$), the formula expresses the identity
        \begin{equation}
            E=m
        \end{equation}

        Subscripts in math mode are written as $a_b$ and superscripts are written as $a^b$. These can be combined and nested to write expressions such as

        \[ T^{i_1 i_2 \dots i_p}_{j_1 j_2 \dots j_q} = T(x^{i_1},\dots,x^{i_p},e_{j_1},\dots,e_{j_q}) \]

        We write integrals using $\int$ and fractions using $\frac{a}{b}$. Limits are placed on integrals using superscripts and subscripts:

        \[ \int_0^1 \frac{dx}{e^x} =  \frac{e-1}{e} \]

        Lower case Greek letters are written as $\omega$ $\delta$ etc. while upper case Greek letters are written as $\Omega$ $\Delta$.

        Mathematical operators are prefixed with a backslash as $\sin(\beta)$, $\cos(\alpha)$, $\log(x)$ etc.

    \section{Tables}
        \begin{center}
            \begin{tabular}{c c c}       % three columns and the text inside each one must be centred. You can also use r to right-align the text and l to left-align it.
                cell1 & cell2 & cell3 \\ % The alignment symbol & is used to demarcate individual table cells within a table row.
                cell4 & cell5 & cell6 \\
                cell7 & cell8 & cell9
            \end{tabular}
        \end{center}

        Table \ref{table:data} shows how to add a table caption and reference a table.

        \begin{table}[h!]
            \centering
            \begin{tabular}{||c|c|c|c||}             % to add vertical rules, between columns, use the vertical line parameter |
                \hline                               % to add horizontal rules, above and below rows, use the \hline command
                Col1 & Col2 & Col2 & Col3 \\ [0.5ex]
                \hline\hline
                1 & 6 & 87837 & 787 \\
                \hline
                2 & 7 & 78 & 5415 \\
                \hline
                3 & 545 & 778 & 7507 \\
                \hline
                4 & 545 & 18744 & 7560 \\
                \hline
                5 & 88 & 788 & 6344 \\ [1ex]
                \hline
            \end{tabular}
            \caption{Table to test captions and labels.}
            \label{table:data}
        \end{table}

    \section{Colors}
        \begin{itemize}
            \color{ForestGreen}
            \item First item
            \item Second item
        \end{itemize}

        \noindent
        {\color{RubineRed} \rule{\linewidth}{0.5mm}}

        The background color of text can also be \textcolor{red}{easily} set. For
        instance, you can change use an \colorbox{BurntOrange}{orange background} and then continue typing.

        \begin{enumerate}
            \item \textcolor{mypink1}{Pink with rgb}
            \item \textcolor{mypink2}{Pink with RGB}
            \item \textcolor{mypink3}{Pink with cmyk}
            \item \textcolor{mygray}{Gray with gray}
        \end{enumerate}

    \section{Chinese}
        \begin{CJK*}{UTF8}{bkai}
            這是一句話。
        \end{CJK*}

    \section{section}
        \subsection{subsection}
            \subsubsection{subsubsection}
                \paragraph{paragraph}
                    \subparagraph{subparagraph}

    \section*{Unnumbered Section}
    \addcontentsline{toc}{section}{Unnumbered Section}
        Text

    \section{Another Section}
        Text

    \bibliographystyle{plain}
    \bibliography{ref}

\end{document}
