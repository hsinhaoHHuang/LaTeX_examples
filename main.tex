
\documentclass[12pt, a4paper]{article} % declares the document type, class

\usepackage{graphicx}
\graphicspath{ {images/} }                                 %configuring the graphicx package

\title{My first LaTeX document}                            % the document title
\author{Hsinhao Huang\thanks{hsinaohuang@gmail.com}}       % here you write the name of the author(s)
\date{\today}                                              % or enter the date manually

\begin{document} % the body of the document

    \maketitle

    First document. This is a simple example, with no
    extra parameters or packages included.

    Some of the \textbf{greatest}                          % bold text in LaTeX is typeset using the \textbf{...} command.
    discoveries in \underline{science}                     % italicised text is produced using the \textit{...} command.
    were made by \textbf{\textit{accident}}.               %  to underline text use the \underline{...} command.

    Some of the greatest \emph{discoveries} in science
    were made by accident.

    \textit{Some of the greatest \emph{discoveries}        % Inside normal text, the emphasized text is italicized, but this behaviour is reversed if used inside an italicized text
    in science were made by accident.}

    \textbf{Some of the greatest \emph{discoveries}
    in science were made by accident.}

    \begin{figure}[h]
        \centering
        \includegraphics[width=0.75\textwidth]{fig_test}   % set the figure’s width to 75% of the text width
        \caption
        {
            A nice plot.
        }
        \label{fig:test}
    \end{figure}

    As you can see in figure \ref{fig:test},
    the function grows near the origin.
    This example is on page \pageref{fig:test}.

    \begin{itemize}                                        % unordered list
        \item The individual entries are indicated with
              a black dot, a so-called bullet.
        \item The text in the entries may be of any length.
    \end{itemize}

    \begin{enumerate}                                      % ordered list
        \item This is the first entry in our list.
        \item The list numbers increase with each entry we add.
    \end{enumerate}

\end{document}
